\documentclass[12pt,onecolumn,letterpaper]{article}
\usepackage{hyperref}
\setcounter{secnumdepth}{5}

\title{dB Cafe Operations Manual}
\author{HKN \& IEEE}
\date{ }

\begin{document}
\maketitle
\tableofcontents
\newpage

\section{Overview}
The dB Cafe is a food establishment operated by the Eta Kappa Nu (HKN) and the Institue of Electrical and Electronics Engineers (IEEE) student orginizations at the University of Michigan in Ann Arbor, Michigan. The dB Cafe is located in the main lobby of the Electrical Engineering and Computer Science (EECS) building on North Campus and open for buisness during the Fall and Winter semesters of the acedemic year.
% ------------------------------------------------------------------------------
% ------------------------------------------------------------------------------
\section{Access to the dB Cafe}
The door to the dB Cafe will remain locked unless the space is being used for activities related to the dB Cafe.

Keys to the dB Cafe are stored/kept in the following locations:
\begin{enumerate}
\item Inside of the HKN Office
\item Inside of the IEEE Office
\item On the HKN Operations Officer's person
\item On the HKN President's person
\end{enumerate}

% ------------------------------------------------------------------------------
% ------------------------------------------------------------------------------
\section{General Shifts: 8:30am - 4:30pm}
% ------------------------------------------------------------------------------
\subsection{Before}
The following actions must be performed by any individual that is working a shift at the dB Cafe before starting their shift.
\begin{enumerate}
\item Arrive to your designated shift on time (not Michigan time)
\item Personal belongings must be place out of sight and not obstructing the working area
\item If you have long hair, use a hair tie or band to put your hair up. You may use a personal hair band or one of the ones provided
\item Wash your hands using the provided soap in the front sink following the pasted guidelines
\end{enumerate}
% ------------------------------------------------------------------------------
\subsection{During}
\subsubsection{Dos}
\begin{enumerate}
\item Use common sense
\item If you are unsure of something ask someone in 1226 EECS (the bullpen)
\item Keep the dB Cafe clean and organized
\item If it looks like trash, throw it out! (use the big trash bin)
\item Use the speakers to play music
\end{enumerate}
\subsubsection{Don'ts}
\begin{enumerate}
\item Eat or drink
\item Do homework
\item Dump anything down the front sink
\item Exchange money for change (we need our quarters)
\item Accept foreign money
\item Refund money used in vending machines
\item Accept tips
\end{enumerate}
\subsubsection{Sales Scanning}
All items sold at the dB Cafe must be tracked through the inventory system. 
\begin{enumerate}
\item When an item is sold, scan the barcode on the item with the provided barcode scanner
\item Some items do not have a barcode or the barcode is difficult to scan, in these cases, scan the custom printed barcode on the documents next to the scanner
\end{enumerate}
\subsubsection{Restocking}
Restock during down times throughout your shift and at the end of your shift. \textbf{Do not leave without restocking!}

When restocking a refrigerator restock from back to front. Place the new items in the back of the refrigerator while bringing older items to the front. This ensures that we sell older products before newer products.
\begin{description}
\item [Front Refrigerator] Use the pail to restock the front refrigerator with items from the back refrigerator
\item [Back Refrigerator] Restock the back refrigerator with items from the back shelves
\item [Coffee Accessories] Restock the lids, stir sticks, and creamers with items from the cabinets. If the cabinets are out of the item, check the back of the dB Cafe
\item [Glass Showcase] \textbf{Do not restock muffins or cookies.} Additional chips can be restocked from the back of the dB Cafe
\item [Candy Drawers] Restock the candy drawers from the candy boxes underneath the glass showcase 
\end{description}
% ------------------------------------------------------------------------------
\subsection{After}
\begin{enumerate}
\item Wait until replacement workers arrive before leaving, do not leave the dB Cafe unattended
\item If no replacement arrives and you need to leave, ask someone in 1226 EECS (the bullpen) for help
\end{enumerate}
% ------------------------------------------------------------------------------
% ------------------------------------------------------------------------------
\section{Special Shifts}
% ------------------------------------------------------------------------------
\subsection{Pizza Shifts: 11:30am - 2:30pm}
The following actions must be performed by any individual that is working a Pizza shift at the dB Cafe.
\subsubsection{Setting Up Pizza Boxes}
\begin{enumerate}
\item When pizza is delivered, place the the pizza bags on the back shelf and pay the delivery person. 
	\begin{enumerate}
	\item If the delivery person tells you the amount owed including tip, pay them that amount
	\item If the delivery person tells you the amount owed \textbf{not} including tip, then pay them the amount owed + 10\% tip rounded up to the nearest 5 dollars.
	\end{enumerate}
\item Using a marker from the money drawer, label each box with the initial corresponding to the type of pizza. This makes it easier to tell which type of pizza is in each box
\item Using pizza boxes from previous shifts and the pizza boxes that just arrived, ensure that there is exactly:
	\begin{enumerate}
	\item one box of Cheese, Mushroom, Spinach Alfredo, and the Weekly Special pizza on top of the shelf above the back counter
	\item one box of Pepperoni, Bacon, Sausage, and BBQ Chicken pizza on top of the pizza bags on the back counter
	\end{enumerate}
\item Close the pizza bags when you are finished moving pizza boxes
\end{enumerate}
\subsubsection{Selling Pizza}
\begin{enumerate}
\item Take a plate and napkin over to the pizza boxes
\item Use the napkin to remove the pizza from the box and put it on the plate.
\item \textbf{Never touch the pizza with your hands}
\item Make sure to close the pizza box when you are finished
\item If a pizza box runs out of slices:
	\begin{enumerate}
	\item Stack the pizza boxes by the door of the dB Cafe
	\item Do not stack pizza boxes above the sign that reads ``Do not Stack Boxes Above This Point''
	\item Take a replacement pizza box out of the pizza bag and put it on the shelf
	\item If there is not a replacement pizza, using a dry erase marker from the money drawer cross off the type of pizza from the menu
	\end{enumerate}
\end{enumerate}
% ------------------------------------------------------------------------------
\subsection{Opening Shift: 8:00am - 8:30am}
The following actions must be performed by any individual that is opening the dB Cafe.

Steps 1 \& 2 must be performed before any of the remaining steps.
\begin{enumerate}
\item Plug in the coffee maker
\item Retrieve the toaster slide from the drying rack in the back of the dB Cafe, attach it to the toaster, and plug in and turn on the toaster 
\item Retrieve the bags of bagels from outside of the dB Cafe door, roll down the bags, and place the bags in/above the storage bin with the corresponding bagel flavor label
\item Retrieve muffin boxes (1 of each type of muffin) from the back shelves, and place them in the display case
\item Retrieve the bagel cutters, coffee pots, coffee scoop, and coffee maker funnel from the drying rack in the back room.
\item Take 3 rolls of quarters and the stack of bills out of the safe in the back room, and neatly place them in the money drawer in the front.
\item Turn on the dB Cafe neon sign and ring light
\item Unlock the locks on both sides of the gate, and open the gate at 8:30am sharp
\end{enumerate}
% ------------------------------------------------------------------------------
\subsection{Closing Shift: 4:30pm - 5:00pm}
The following actions must be performed by any individual that is closing the dB Cafe.

\begin{enumerate}
\item Begin filling the sinks with (hot) water 
\item Wipe down front counter with sanitation wipes, clear path for the gate
\item Close the gate and lock the latches at each end
\item Turn off the dB Cafe neon sign and ring light
\item Unplug the coffee maker and toaster
\item Bring leftover pizza and bagels to 1226 EECS (the bullpen)
\item Offer leftover coffee to students in 1226 EECS (the bullpen), dump extra coffee down the floor drain underneath the back sinks
\item Finish wiping down all the counters, coffee accessories organizer, and any dirty/dusty surfaces
\item Follow the directions above the back sinks to prep the washing station
\item Wash the coffee pots, coffee filter, coffee scoop, toaster slide, toaster crumb tray, and bagel slicers
\item Move the money from the front cash drawer to the back safe, leaving \$15 in ones and all quarters in the front cash drawer
\item Restock the refridgerators, candy drawers, chips box, and all condiments and utensils
\item Turn off all the lights and lock the door when you leave
\end{enumerate}
% ------------------------------------------------------------------------------
% ------------------------------------------------------------------------------
\section{Food Serving Guidelines}
% ------------------------------------------------------------------------------
\subsection{Coffee}
\begin{enumerate}
\item Remove the coffee maker funnel from the coffee maker and dump the old coffee contents in the large trash bin. While doing so, hold a paper plate beneath the funnel to prevent drips from getting on the floor
\item Lay the paper plate on the counter and place the funnel on top of the it.
\item Remove a single coffee filter from the stack next to the coffee maker and place into the now empty funnel
\item Open the coffee grounds container and using the enclosed scoop, place one scoop of coffee grounds in the coffee funnel inside of the coffee filter
\item Put the now full coffee funnel back into the coffee maker
\item Place an empty coffee pot underneath the coffee maker
\item Press the brew button
\item Wain until the screen on the coffee maker says ``Ready'' before removing the coffee pot from underneath the coffee maker  
\end{enumerate}
% ------------------------------------------------------------------------------
\subsection{Tea \& Hot Chocolate}
\begin{enumerate}
\item Fill a small cup with hot water from the hot water spigot on the coffee maker
\item If serving hot chocolate, hand the customer the full cup of hot water and a hot chocolate packet. \textbf{Do Not Open the Hot Chocolate Packet Yourself}
\item If serving tea, hand the customer the full cup of how water and allow the customer to select a single tea bag from the assorted tea bag container
\end{enumerate}
% ------------------------------------------------------------------------------
\subsection{Ramen}

% ------------------------------------------------------------------------------
% ------------------------------------------------------------------------------
\section{Treasurer Duties}
The treasurer of HKN is prohibited from accepting money from customers because they are responsible for counting the daily revenue. Thus, this individual is prohibited from working a shift at the dB Cafe, excluding the opening or closing shift because no sales occur during these shifts.
% ------------------------------------------------------------------------------
\subsection{Daily money counting}
The following actions must be preformed by the treasurer of HKN at 4:30pm daily. 
\begin{enumerate}
\item To determine the day's revenue, count all of the bills in the money drawer and subtract the daily starting amount (\$850 = 15x\$20, 30x\$10, 40x\$5, 50x\$1)
\item Enter the day's revenue in th dB Cafe revenue spreadsheet 
\item Using previous days revenue and the current days revenue, create stacks of 25 of each denomination of bills (stacks of 50 for \$1s), and create a ``leftover stack'' with all excess bills 
\item Place the daily revenue into the revenue folder which is stored in the HKN safe
\item Place the daily starting amount into the dB Cafe safe to be used the next time the dB Cafe is open for business
\end{enumerate}
% ------------------------------------------------------------------------------
\subsection{Weekly bank deposits}
The following actions must be preformed by the treasurer of HKN on a weekly basis.
\begin{enumerate}
\item Take the weekly dB Cafe revenue money from the HKN safe to the UM Credit Union
\item Deposit all \$100 and \$50 bills and all stacks of 25 \$20 bills into the dB Cafe bank account
\item Using the remaining revenue money, deposit and/or withdraw to possess 100 x \$1 + 50 x \$5 + 100 x \$10. Do not preform deposits using the ``leftover bills stack'', only deposit or withdraw using stacks of 25 (or stacks of 50 for \$1s)
\item To deposit, write up a deposit slip with the account number and total bill counts for both deposits and withdrawals.
\end{enumerate}

% ------------------------------------------------------------------------------
\subsection{Monthly Taxes}
\begin{enumerate}
\item Print off a 5080 form (Sales, Use, and Withholding Taxes) from \url{http://www.michigan.gov/taxes/0,4676,7-238-44079-344959--,00.html}
\item Using the dB Café revenue spreadsheet, input the total revenue into the first box, then proceed to follow the instructions.
\item If completing and filing the 5080 before the 12th of the month, subtract \$6 from the total
\item Write a check for the final tax amount (0.06 * Total revenue - (\$6 if filing before the 12th)) to the ``State of Michigan'' and mail, following the instructions on the bottom of the 5080 form
\end{enumerate}
% ------------------------------------------------------------------------------
% ------------------------------------------------------------------------------
\section{Restocking the dB Cafe}
\begin{enumerate}
\item When not in use, a checkbook for the dB Cafe bank account is kept inside of the dB Cafe safe
\item Use the dB Cafe checkbook to purchase inventory to restock the dB Cafe
\item Place all receipts for purchases of inventory in the dB Cafe safe for use by the treasurer for accounting purposes
\end{enumerate}
\end{document}
